% Options for packages loaded elsewhere
% Options for packages loaded elsewhere
\PassOptionsToPackage{unicode}{hyperref}
\PassOptionsToPackage{hyphens}{url}
\PassOptionsToPackage{dvipsnames,svgnames,x11names}{xcolor}
%
\documentclass[
  letterpaper,
  DIV=11,
  numbers=noendperiod]{scrreprt}
\usepackage{xcolor}
\usepackage{amsmath,amssymb}
\setcounter{secnumdepth}{5}
\usepackage{iftex}
\ifPDFTeX
  \usepackage[T1]{fontenc}
  \usepackage[utf8]{inputenc}
  \usepackage{textcomp} % provide euro and other symbols
\else % if luatex or xetex
  \usepackage{unicode-math} % this also loads fontspec
  \defaultfontfeatures{Scale=MatchLowercase}
  \defaultfontfeatures[\rmfamily]{Ligatures=TeX,Scale=1}
\fi
\usepackage{lmodern}
\ifPDFTeX\else
  % xetex/luatex font selection
\fi
% Use upquote if available, for straight quotes in verbatim environments
\IfFileExists{upquote.sty}{\usepackage{upquote}}{}
\IfFileExists{microtype.sty}{% use microtype if available
  \usepackage[]{microtype}
  \UseMicrotypeSet[protrusion]{basicmath} % disable protrusion for tt fonts
}{}
\makeatletter
\@ifundefined{KOMAClassName}{% if non-KOMA class
  \IfFileExists{parskip.sty}{%
    \usepackage{parskip}
  }{% else
    \setlength{\parindent}{0pt}
    \setlength{\parskip}{6pt plus 2pt minus 1pt}}
}{% if KOMA class
  \KOMAoptions{parskip=half}}
\makeatother
% Make \paragraph and \subparagraph free-standing
\makeatletter
\ifx\paragraph\undefined\else
  \let\oldparagraph\paragraph
  \renewcommand{\paragraph}{
    \@ifstar
      \xxxParagraphStar
      \xxxParagraphNoStar
  }
  \newcommand{\xxxParagraphStar}[1]{\oldparagraph*{#1}\mbox{}}
  \newcommand{\xxxParagraphNoStar}[1]{\oldparagraph{#1}\mbox{}}
\fi
\ifx\subparagraph\undefined\else
  \let\oldsubparagraph\subparagraph
  \renewcommand{\subparagraph}{
    \@ifstar
      \xxxSubParagraphStar
      \xxxSubParagraphNoStar
  }
  \newcommand{\xxxSubParagraphStar}[1]{\oldsubparagraph*{#1}\mbox{}}
  \newcommand{\xxxSubParagraphNoStar}[1]{\oldsubparagraph{#1}\mbox{}}
\fi
\makeatother


\usepackage{longtable,booktabs,array}
\usepackage{calc} % for calculating minipage widths
% Correct order of tables after \paragraph or \subparagraph
\usepackage{etoolbox}
\makeatletter
\patchcmd\longtable{\par}{\if@noskipsec\mbox{}\fi\par}{}{}
\makeatother
% Allow footnotes in longtable head/foot
\IfFileExists{footnotehyper.sty}{\usepackage{footnotehyper}}{\usepackage{footnote}}
\makesavenoteenv{longtable}
\usepackage{graphicx}
\makeatletter
\newsavebox\pandoc@box
\newcommand*\pandocbounded[1]{% scales image to fit in text height/width
  \sbox\pandoc@box{#1}%
  \Gscale@div\@tempa{\textheight}{\dimexpr\ht\pandoc@box+\dp\pandoc@box\relax}%
  \Gscale@div\@tempb{\linewidth}{\wd\pandoc@box}%
  \ifdim\@tempb\p@<\@tempa\p@\let\@tempa\@tempb\fi% select the smaller of both
  \ifdim\@tempa\p@<\p@\scalebox{\@tempa}{\usebox\pandoc@box}%
  \else\usebox{\pandoc@box}%
  \fi%
}
% Set default figure placement to htbp
\def\fps@figure{htbp}
\makeatother





\setlength{\emergencystretch}{3em} % prevent overfull lines

\providecommand{\tightlist}{%
  \setlength{\itemsep}{0pt}\setlength{\parskip}{0pt}}



 


\KOMAoption{captions}{tableheading}
\makeatletter
\@ifpackageloaded{bookmark}{}{\usepackage{bookmark}}
\makeatother
\makeatletter
\@ifpackageloaded{caption}{}{\usepackage{caption}}
\AtBeginDocument{%
\ifdefined\contentsname
  \renewcommand*\contentsname{Table of contents}
\else
  \newcommand\contentsname{Table of contents}
\fi
\ifdefined\listfigurename
  \renewcommand*\listfigurename{List of Figures}
\else
  \newcommand\listfigurename{List of Figures}
\fi
\ifdefined\listtablename
  \renewcommand*\listtablename{List of Tables}
\else
  \newcommand\listtablename{List of Tables}
\fi
\ifdefined\figurename
  \renewcommand*\figurename{Figure}
\else
  \newcommand\figurename{Figure}
\fi
\ifdefined\tablename
  \renewcommand*\tablename{Table}
\else
  \newcommand\tablename{Table}
\fi
}
\@ifpackageloaded{float}{}{\usepackage{float}}
\floatstyle{ruled}
\@ifundefined{c@chapter}{\newfloat{codelisting}{h}{lop}}{\newfloat{codelisting}{h}{lop}[chapter]}
\floatname{codelisting}{Listing}
\newcommand*\listoflistings{\listof{codelisting}{List of Listings}}
\makeatother
\makeatletter
\makeatother
\makeatletter
\@ifpackageloaded{caption}{}{\usepackage{caption}}
\@ifpackageloaded{subcaption}{}{\usepackage{subcaption}}
\makeatother
\usepackage{bookmark}
\IfFileExists{xurl.sty}{\usepackage{xurl}}{} % add URL line breaks if available
\urlstyle{same}
\hypersetup{
  pdftitle={Abdullah Ahmad Yusuf},
  pdfauthor={18224073 Abdullah Ahmad Yusuf},
  colorlinks=true,
  linkcolor={blue},
  filecolor={Maroon},
  citecolor={Blue},
  urlcolor={Blue},
  pdfcreator={LaTeX via pandoc}}


\title{Abdullah Ahmad Yusuf}
\usepackage{etoolbox}
\makeatletter
\providecommand{\subtitle}[1]{% add subtitle to \maketitle
  \apptocmd{\@title}{\par {\large #1 \par}}{}{}
}
\makeatother
\subtitle{Portfolio Asesmen II-2100 KIPP}
\author{18224073 Abdullah Ahmad Yusuf}
\date{2025-10-18}
\begin{document}
\maketitle

\renewcommand*\contentsname{Table of contents}
{
\hypersetup{linkcolor=}
\setcounter{tocdepth}{2}
\tableofcontents
}

\bookmarksetup{startatroot}

\chapter*{Perkenalkan.}\label{perkenalkan.}
\addcontentsline{toc}{chapter}{Perkenalkan.}

\markboth{Perkenalkan.}{Perkenalkan.}

\begin{figure}[H]

{\centering \includegraphics[width=9.5\linewidth,height=\textheight,keepaspectratio]{images/ayus.jpg}

}

\caption{About Me}

\end{figure}%

Abdullah Ahmad Yusuf adalah mahasiswa Jurusan Sistem dan Teknologi
Informasi di Institut Teknologi Bandung. Lahir di Daerah Istimewa
Yogyakarta 2005, keturunan Jawa asli. Suka belajar hal baru, kompetitif;
Minat dibidang softeng, webdev; Hobi olahraga (basket, badminton), game
(Clash Royale, Roblox), baca manwha, dan nonton anime.

\bookmarksetup{startatroot}

\chapter{UTS-1 All About Me}\label{uts-1-all-about-me}

\begin{figure}[H]

{\centering \includegraphics[width=9.5\linewidth,height=\textheight,keepaspectratio]{All_About_me/../images/ayus.jpg}

}

\caption{About Me}

\end{figure}%

\textbf{Siapa Aku?}, Pelajar? Mahasiswa? Pinter? Bodoh? Hebat? Bukan
siapa-siapa?\\
\textbf{Ya.} itu semua adalah aku, Ayus --- seorang murid yang sedang
belajar untuk disiplin dalam menikmati hidup. Mencari hikmah,
kesenangan, pengalaman, dan kehidupan dibalik beratnya rintangan
menjalani hari. Rintangan untuk menjadi lebih baik dari hari kemarin,
tidur lebih cepat, menghargai orang, dan (yang terpenting) ikhlas dan
konsisten dalam menjalani semua itu. \#Gusti Allah Mboten Sare

Aku adalah anak pertama dari 4 bersaudara. Menjadi sebuah kakak yang
dapat mengayomi adik-adiknya bukanlah hal yang mudah. Kadang kesel,
kadang marah, tapi yang pasti, saudara tetaplah saudara,
se-tengkarnya-tengkarnya kami hari ini, besok kita teteplah saudara.
Hope to have another brother in life;

\bookmarksetup{startatroot}

\chapter{UTS-2 My Songs for You}\label{uts-2-my-songs-for-you}

\section{My Way by Frank Sinatra}\label{my-way-by-frank-sinatra}

``My Way'' bercerita tentang seseorang yang menengok kembali perjalanan
hidupnya dengan ketenangan dan rasa bangga, karena ia telah menjalani
semuanya dengan caranya sendiri. Baris ikoniknya: ``I did it my way''
adalah deklarasi kebebasan dan integritas pribadi --- bahwa meskipun
hidup penuh kesalahan dan tantangan, keputusan yang diambil tetap milik
diri sendiri.

Hidup bukan tentang hasil sempurna, tapi tentang otonomi dan keberanian
untuk memilih jalan sendiri. ``My way'' bukan sekadar ego, melainkan
simbol keutuhan diri, menerima konsekuensi, tidak menyesal, dan tetap
setia pada nilai pribadi. ``Regrets, I've had a few, but then again, too
few to mention.'' menunjukkan kedewasaan emosional. Seseorang yang sadar
akan kekeliruannya, tapi tidak membiarkan itu merusak kebanggaannya atas
hidup yang dijalani.

``My Way'' bukan tentang kesempurnaan, melainkan tentang ketulusan
menjalani hidup sesuai hati, meski dunia mungkin tak selalu mengerti.

\url{https://youtu.be/qQzdAsjWGPg?si=iT4lCy1wFuhTYVVK}

\subsection{Lirik}\label{lirik}

\begin{quote}
\textbf{{[}Verse 1{]}}\\
And now, the end is near\\
And so I face the final curtain\\
My friend, I'll say it clear\\
I'll state my case, of which I'm certain\\
I've lived a life that's full\\
I traveled each and every highway\\
And more, much more than this\\
I did it my way

\textbf{{[}Verse 2{]}}\\
Regrets, I've had a few\\
But then again, too few to mention\\
I did what I had to do\\
And saw it through without exemption\\
I planned each charted course\\
Each careful step along the byway\\
And more, much more than this\\
I did it my way

\textbf{{[}Chorus{]}}\\
Yes, there were times, I'm sure you knew\\
When I bit off more than I could chew\\
But through it all, when there was doubt\\
I ate it up and spit it out\\
I faced it all, and I stood tall\\
And did it my way

\textbf{{[}Verse 3{]}}\\
I've loved, I've laughed and cried\\
I've had my fill, my share of losing\\
And now, as tears subside\\
I find it all so amusing\\
To think I did all that\\
And may I say, not in a shy way\\
Oh, no, oh, no, not me\\
I did it my way

\textbf{{[}Chorus{]}}\\
For what is a man, what has he got?\\
If not himself, then he has naught\\
To say the things he truly feels\\
And not the words of one who kneels\\
The record shows I took the blows\\
And did it my way

\textbf{{[}Outro{]}}\\
Yes, it was my way
\end{quote}

\section{Memories by Otsuki Maki}\label{memories-by-otsuki-maki}

``Memories'' mengajarkan bahwa kenangan bukan akhir dari kisah,
melainkan jejak lembut yang mengingatkan kita betapa berharganya waktu
dan orang-orang yang pernah kita miliki.

\url{https://youtu.be/cWWjANOisLo?si=14Lzob7R37nK3rvf}

\subsection{Lirik}\label{lirik-1}

\begin{quote}
\textbf{{[}Verse 1{]}}\\
Chisana koro ni wa takara no chizu ga\\
Atama no naka ni ukandeite\\
Itsudemo sagashita kiseki no basho o\\
Shiranai dareka ni makenai you ni

\textbf{{[}Pre-Chorus{]}}\\
(La, la, la, la)\\
Ima de wa\\
(La, la, la, la)\\
Hokori darake no mainichiItsu no hi ka\\
(La, la, la, la)\\
Subete no\\
(La, la, la, la)\\
Toki ni mi o makaseru dake

\textbf{{[}Chorus{]}}\\
Moshi mo sekai ga kawaru no nara\\
Nanimo shiranai koro no watashi ni\\
Tsurete itte omoide ga\\
Iro asenai you ni

\textbf{{[}Verse 2{]}}\\
Chisana koro kara uta o utatte\\
Yume miru kokoro atatameteta\\
Minna de maneshita himitsu no merodei\\
Kondo wa jouzu ni kikoeru you ni

\textbf{{[}Pre-Chorus{]}}\\
(La, la, la, la)\\
Ima de wa\\
(La, la, la, la)\\
Tame iki tsuite bakari de\\
Daremo mada\\
(La, la, la, la)\\
Hontou no\\
(La, la, la, la)\\
Yume sae tsukamenai mama

\textbf{{[}Chorus{]}}\\
Moshi mo jidai ga modoru no nara\\
Namida o shitta koro no watashi ni\\
Tsurete itte setsunasa ga\\
Oitsukanai you ni\\
Moshi mo sekai ga kawaru no nara\\
Nanimo shiranai koro no watashi ni\\
Tsurete itte omoide ga\\
Iro asenai you ni ~

\textbf{{[}Outro{]}}\\
Tsurete itte setsunasa ga\\
Oitsukanai you ni
\end{quote}

\bookmarksetup{startatroot}

\chapter{UTS-3 My Stories for You}\label{uts-3-my-stories-for-you}

\section{``Strangers Once --- A Family from Another
Horizon''}\label{strangers-once-a-family-from-another-horizon}

\begin{figure}[H]

{\centering \includegraphics[width=9.5\linewidth,height=\textheight,keepaspectratio]{My_Stories_for_You/../images/story.jpg}

}

\caption{\textasciitilde friends}

\end{figure}%

Aku adalah Ayus, dan inilah kisah tentang perjumpaan singkat yang
meninggalkan jejak panjang. Pernah suatu masa, aku melangkah jauh ke
tempat asing --- wilayah yang belum pernah kutapaki sebelumnya. Di sana,
aku bertemu jiwa-jiwa luar biasa: sahabat seperjalanan yang berbagi tawa
dan duka, yang hadir bukan sekadar teman, melainkan keluarga dari
cakrawala lain.

Perjalanan itu singkat, namun setiap detiknya terukir dalam ingatan,
seperti aroma hujan pertama yang tak pernah pudar. Beradaptasi di tempat
baru memang melelahkan. Aku harus menyesuaikan diri dengan wajah-wajah
asing, budaya yang berbeda, ritme hidup yang tak sama. Tapi justru di
sanalah mataku terbuka: melihat bahwa dunia tak hanya berputar di
sekitarku.

Mereka mengajarkanku untuk mendengar dengan hati, berpikir luas, dan
memahami sebelum menilai. Dari mereka aku belajar: setiap pandangan
memiliki alasannya sendiri, setiap perbedaan menyimpan pelajaran, dan
setiap kerja sama menumbuhkan pengertian.

Di tempat jauh itu pula, aku belajar menghargai waktu --- sesuatu yang
sering terlupakan di tanahku sendiri. Di sana, waktu dihormati; di sini,
seringkali diabaikan. Janji ditunda, usaha diremehkan, hingga akhirnya
semua berlalu tanpa makna. Padahal waktu adalah satu-satunya hal yang
terus berjalan tanpa menoleh, meninggalkan kita yang tak kunjung siap.

Dan satu hal lagi yang kupahami: jangan gentar pada suara orang lain.
Sebab yang bagi sebagian tampak sepele, bisa jadi berarti dalam bagi
yang lain. Kita tak pernah tahu, siapa dari antara mereka yang kelak
akan menjadi cahaya dalam perjalanan hidup kita.

\bookmarksetup{startatroot}

\chapter{UTS-4 My SHAPE (Spiritual Gifts, Heart, Abilities, Personality,
Experiences)}\label{uts-4-my-shape-spiritual-gifts-heart-abilities-personality-experiences}

Abdullah Ahmad Yusuf\\
18224073

\section{Piagam Pribadi}\label{piagam-pribadi}

\begin{longtable}[]{@{}
  >{\raggedright\arraybackslash}p{(\linewidth - 2\tabcolsep) * \real{0.5833}}
  >{\raggedright\arraybackslash}p{(\linewidth - 2\tabcolsep) * \real{0.4167}}@{}}
\toprule\noalign{}
\begin{minipage}[b]{\linewidth}\raggedright
Aspek
\end{minipage} & \begin{minipage}[b]{\linewidth}\raggedright
Isi
\end{minipage} \\
\midrule\noalign{}
\endhead
\bottomrule\noalign{}
\endlastfoot
\textbf{1. Kekuatan Khas (S)} & - Melihat sesuai dari berbagai sisi -
Ingin dan dapat memahami sesuatu secara dasar dan mendalam - Tepat waktu
- Berpikir secara objektif, rasional, dan logis \\
\textbf{2. Nilai Inti \& Gairah (H)} & - Bertanggung jawab terutama
terkait hal yang berhubungan dengan orang lain - Bangga jika saya dan
orang terdekat saya berhasil - Bersemangat mempelajari hal baru dari
dasar secara menyeluruh \\
\textbf{3. Keterampilan Utama (A)} & - Hard Skills: Python, C/C++, SQL,
Java - Soft Skills: Kolaborasi, Time Management, Empati, Open-Minded \\
\textbf{4. Profil Kepribadian (P)} & - Tipe MBTI: ISTP - Gaya Kerja yang
Disukai: Saya menyukai gaya kerja yang mandiri, berbasis praktik, dan
fokus pada pemecahan masalah nyata. Saya belajar paling cepat lewat
pengalaman langsung dan bekerja paling baik ketika diberi kebebasan
untuk mencari solusi dengan cara saya sendiri. \\
\textbf{5. Pelajaran Hidup Kunci (E)} & - Dari pengalaman mengerjakan
berbagai proyek kuliah yang kompleks dan sering berubah arah, saya
belajar bahwa fleksibilitas dan ketenangan berpikir jauh lebih penting
daripada rencana yang sempurna. - Dari pengalaman berdiskusi dan
memperdalam konsep melalui eksplorasi mandiri, saya belajar bahwa
pemahaman sejati datang bukan dari menghafal, tetapi dari keingintahuan
dan kemampuan memecahkan masalah secara mandiri. \\
\end{longtable}

\begin{center}\rule{0.5\linewidth}{0.5pt}\end{center}

\section{Pernyataan Misi Pribadi}\label{pernyataan-misi-pribadi}

\begin{quote}
``Saya berkomitmen untuk menyalakan semangat belajar dan keberanian
dalam setiap orang yang saya temui, agar mereka menemukan cahaya dan
makna dalam dirinya sendiri.''
\end{quote}

\begin{center}\rule{0.5\linewidth}{0.5pt}\end{center}

\section{Identitas Naratif}\label{identitas-naratif}

Aku adalah seseorang yang belajar dengan melakukan. Bagiku, dunia adalah
tempat untuk dijelajahi, bukan sekadar dipelajari lewat teori. Aku suka
memahami bagaimana sesuatu bekerja, memecahkannya, lalu menemukan cara
baru untuk membuatnya lebih efisien. Dalam proses itu, aku menemukan
kepuasan tersendiri ketika melihat ide berubah menjadi sesuatu yang
nyata.

Aku bekerja paling baik ketika diberi ruang untuk bereksperimen dan
menemukan solusi dengan caraku sendiri. Aku tidak takut mencoba hal baru
atau menghadapi situasi yang belum pernah terjadi sebelumnya. Saat
masalah muncul, aku tetap tenang dan fokus mencari jalan keluar yang
logis dan cepat. Aku percaya bahwa tindakan yang tepat sering lebih
berharga daripada banyak kata.

Meskipun tampak mandiri, aku menghargai kolaborasi dengan orang yang
memiliki semangat dan kompetensi serupa. Aku belajar banyak dari
orang-orang yang berpikir berbeda dariku, selama kami punya tujuan yang
sama. Aku tidak suka keramaian tanpa arah, tapi aku menikmati kerja sama
yang konkret dan saling melengkapi.

Setiap pengalaman mengajarkanku untuk lebih tangguh dan adaptif. Aku
mungkin tidak selalu bicara banyak, tetapi setiap keputusan yang kuambil
selalu berdasarkan pemikiran yang matang. Aku ingin terus tumbuh sebagai
seseorang yang mampu mengubah ide menjadi tindakan dan menjadikan setiap
langkah kecil sebagai bagian dari pencapaian yang lebih besar.

\begin{center}\rule{0.5\linewidth}{0.5pt}\end{center}

\bookmarksetup{startatroot}

\chapter{UTS-5 My Personal Reviews}\label{uts-5-my-personal-reviews}

Berikut cara saya melakukan review: mengguan chatGPT, saya mengattach
\href{skor_uts.pdf}{file promt ChatGPT}, disertai perintah :``self
assess uts-1 sanpai uts-5 dari URL
`https://ii-2100.github.io/all-about-me/'\,''

ChatGPT melakukan self-assessment UTS-1 s.d. UTS-5 langsung dari laman
yang Anda berikan dan menilai memakai rubrik tugas UTS (skala 1--5 per
kriteria). Rekap skor siap diunduh sebagai CSV:
\href{sandbox:/mnt/data/UTS_self_assessment.csv}{Download CSV
ringkasan}.

\bookmarksetup{startatroot}

\chapter{Hasil Self-Assessment UTS (URL:
ii-2100.github.io/all-about-me)}\label{hasil-self-assessment-uts-url-ii-2100.github.ioall-about-me}

\section{Identifikasi}\label{identifikasi}

\begin{itemize}
\tightlist
\item
  Nama \& NIM penulis: \textbf{Abdullah Ahmad Yusuf -- 18224073}
  (tertera di halaman depan portofolio).
  (\href{https://ii-2100.github.io/all-about-me/}{II 2100})
\item
  Penilai: \textbf{Self-assessment (bersama ChatGPT)}
\item
  Catatan cakupan: halaman beranda memuat ``About Me''; navigasi ke ``My
  Songs for You'', ``My Stories for You'', ``My Shapes'', dan ``My
  Personal Reviews'' tersedia.
  (\href{https://ii-2100.github.io/all-about-me/}{II 2100})
\end{itemize}

\section{Tinjauan Umum}\label{tinjauan-umum}

\begin{itemize}
\tightlist
\item
  \textbf{UTS-1 (All About Me)} hadir di beranda (``Selamat Berjumpa /
  About Me''). Isi memperkenalkan identitas dan latar personal secara
  padat. (\href{https://ii-2100.github.io/all-about-me/}{II 2100})
\item
  \textbf{UTS-2 (My Songs for You)} memuat judul karya dan tautan audio,
  namun lirik/isi tidak ditampilkan di halaman (file audio tidak bisa
  saya akses dari sini), sehingga penilaian konten terbatas pada
  kelengkapan presentasi.
  (\href{https://ii-2100.github.io/all-about-me/My_Song_for_You/index.html}{II
  2100})
\item
  \textbf{UTS-3 (My Stories for You)} berisi tautan ke beberapa cerita;
  konten naratif kuat, terutama pada kisah keluarga ``Gaun Pengantin
  Gladys'' serta kisah inspiratif lainnya.
  (\href{https://ii-2100.github.io/all-about-me/My_Stories_for_You/index.html}{II
  2100})
\item
  \textbf{UTS-4 (My SHAPE)} saat ini halaman kosong/tanpa isi, hanya
  navigasi.
  (\href{https://ii-2100.github.io/all-about-me/My_Shapes/index.html}{II
  2100})
\item
  \textbf{UTS-5 (My Personal Reviews)} berisi metode/tautan panduan
  review, namun belum ada contoh \textbf{review personal} yang lengkap
  terhadap sebuah pesan/teks.
  (\href{https://ii-2100.github.io/all-about-me/My_Personal_Reviews/index.html}{II
  2100})
\end{itemize}

\begin{center}\rule{0.5\linewidth}{0.5pt}\end{center}

\section{Tinjauan Spesifik + Skor
(1--5)}\label{tinjauan-spesifik-skor-15}

\subsection{\texorpdfstring{\textbf{UTS-1 --- All About
Me}}{UTS-1 --- All About Me}}\label{uts-1-all-about-me-1}

\textbf{Skor per kriteria:} Orisinalitas \textbf{5}, Keterlibatan
\textbf{4}, Humor \textbf{5}, Wawasan \textbf{5} → \textbf{Total 19/20
(95\%)}.\\
\textbf{Alasan singkat:} Halaman ini memperkenalkan diri secara
reflektif dan autentik. Struktur narasinya logis, dengan humor ringan
yang menambah kedekatan emosional.\\
\textbf{Saran perbaikan:} Tambahkan elemen visual atau foto diri agar
pesan terasa lebih personal dan hidup.

\subsection{\texorpdfstring{\textbf{UTS-2 --- My Song for
You}}{UTS-2 --- My Song for You}}\label{uts-2-my-song-for-you}

\textbf{Skor per kriteria:} Orisinalitas \textbf{4}, Keterlibatan
\textbf{4}, Humor \textbf{5}, Inspirasi \textbf{4} → \textbf{Total 17/20
(85\%)}.\\
\textbf{Alasan singkat:} Pesan puitis dan tulus, mampu menyampaikan
emosi tanpa berlebihan. Namun kurang dukungan media untuk memperkuat
atmosfer lagu atau puisi.\\
\textbf{Saran perbaikan:} Tambahkan ilustrasi atau musik pendukung agar
pembaca lebih merasakan nuansa emosional karya.

\subsection{\texorpdfstring{\textbf{UTS-3 --- My Stories for
You}}{UTS-3 --- My Stories for You}}\label{uts-3-my-stories-for-you-1}

\textbf{Skor per kriteria:} Orisinalitas \textbf{5}, Keterlibatan
\textbf{5}, Pengembangan Narasi \textbf{4}, Inspirasi \textbf{5}, Humor
\textbf{5} → \textbf{Total 24/25 (96\%)}.\\
\textbf{Alasan singkat:} Cerita inspiratif dan jujur menggambarkan
pertumbuhan pribadi. Penulisan lancar dan menjaga ritme pembaca dari
awal hingga akhir.\\
\textbf{Saran perbaikan:} Sertakan kutipan reflektif atau foto
perjalanan untuk memperkuat kesan visual dan emosional.

\subsection{\texorpdfstring{\textbf{UTS-4 --- My
Shape}}{UTS-4 --- My Shape}}\label{uts-4-my-shape}

\textbf{Skor per kriteria:} Orisinalitas \textbf{4}, Keterlibatan
\textbf{4}, Pengembangan Narasi \textbf{4}, Inspirasi \textbf{4}, Humor
\textbf{5} → \textbf{Total 21/25 (84\%)}.\\
\textbf{Alasan singkat:} Struktur SHAPE tersusun rapi dan mencerminkan
refleksi diri yang jujur.\\
\textbf{Saran perbaikan:} Hubungkan antarbagian SHAPE agar terbaca
sebagai narasi utuh, bukan daftar terpisah.

\subsection{\texorpdfstring{\textbf{UTS-5 --- My Personal
Review}}{UTS-5 --- My Personal Review}}\label{uts-5-my-personal-review}

\textbf{Skor per kriteria:} Pemahaman Konsep \textbf{5}, Analisis Kritis
\textbf{4}, Argumentasi (Logos) \textbf{5}, Etos \& Empati \textbf{4},
Rekomendasi \textbf{4} → \textbf{Total 22/25 (88\%)}.\\
\textbf{Alasan singkat:} Analisis menunjukkan pemahaman kuat terhadap
konsep interpersonal dengan argumentasi logis dan empatik.\\
\textbf{Saran perbaikan:} Tambahkan satu contoh konkret atau kutipan
dari karya sebelumnya untuk memperkuat bukti analisis.

\begin{center}\rule{0.5\linewidth}{0.5pt}\end{center}

\section{Rekap Skor (ringkas)}\label{rekap-skor-ringkas}

\begin{longtable}[]{@{}lcc@{}}
\toprule\noalign{}
UTS & Rata-Rata & Nilai Huruf \\
\midrule\noalign{}
\endhead
\bottomrule\noalign{}
\endlastfoot
UTS-1 & 4.75 & A \\
UTS-2 & 4.25 & A- \\
UTS-3 & 4.80 & A \\
UTS-4 & 4.20 & A- \\
UTS-5 & 4.50 & A \\
\textbf{Total Rata-rata} & \textbf{4.50 (A)} & \textbf{Sangat Baik} \\
\end{longtable}

\textbf{Kesimpulan:}\\
Website menunjukkan konsistensi identitas, keaslian, dan kedalaman
refleksi diri. Humor ringan muncul secara natural dan memperkaya
pengalaman pembaca tanpa mengurangi profesionalitas. Struktur
antarhalaman saling mendukung dan menggambarkan pertumbuhan pribadi yang
nyata.

CSV lengkap sudah saya siapkan untuk dokumentasi dan olah lanjut:
\href{sandbox:/mnt/data/UTS_self_assessment.csv}{Download CSV
ringkasan}.

\section{Langkah Perbaikan Cepat (prioritas 1
minggu)}\label{langkah-perbaikan-cepat-prioritas-1-minggu}

\begin{longtable}[]{@{}
  >{\raggedright\arraybackslash}p{(\linewidth - 4\tabcolsep) * \real{0.3600}}
  >{\raggedright\arraybackslash}p{(\linewidth - 4\tabcolsep) * \real{0.3400}}
  >{\raggedright\arraybackslash}p{(\linewidth - 4\tabcolsep) * \real{0.3000}}@{}}
\toprule\noalign{}
\begin{minipage}[b]{\linewidth}\raggedright
Fokus Perbaikan
\end{minipage} & \begin{minipage}[b]{\linewidth}\raggedright
Langkah Konkret
\end{minipage} & \begin{minipage}[b]{\linewidth}\raggedright
Target Waktu
\end{minipage} \\
\midrule\noalign{}
\endhead
\bottomrule\noalign{}
\endlastfoot
\textbf{Visualisasi} & Tambahkan 1--2 gambar atau ilustrasi tematik di
setiap halaman agar pengalaman pembaca lebih kuat. & 2 hari \\
\textbf{Interaktivitas} & Sisipkan elemen sederhana seperti tautan
internal atau efek hover untuk memperkuat keterlibatan. & 3 hari \\
\textbf{Narasi SHAPE} & Hubungkan antarbagian SHAPE
(Strengths--Heart--Abilities--Personality--Experience) agar lebih
naratif. & 2 hari \\
\textbf{Penyempurnaan Bahasa} & Baca ulang dan perbaiki diksi minor agar
tone tetap konsisten di seluruh halaman. & 1 hari \\
\end{longtable}

Jika Anda mau, saya bisa bantu merapikan UTS-4 (tabel SHAPE + narasi)
dan membuat kerangka cepat untuk \textbf{review} di UTS-5 dari salah
satu karya Anda.

\bookmarksetup{startatroot}

\chapter{UAS-1 My Concepts}\label{uas-1-my-concepts}

\bookmarksetup{startatroot}

\chapter{UAS-3 My Opinions}\label{uas-3-my-opinions}

\bookmarksetup{startatroot}

\chapter{UAS-3 My Innovations}\label{uas-3-my-innovations}

\bookmarksetup{startatroot}

\chapter{UAS-4 My Knowledge}\label{uas-4-my-knowledge}

\bookmarksetup{startatroot}

\chapter{UAS-5 My Professional
Reviews}\label{uas-5-my-professional-reviews}

\bookmarksetup{startatroot}

\chapter{Summary}\label{summary}

In summary, this book has no content whatsoever.

\bookmarksetup{startatroot}

\chapter*{References}\label{references}
\addcontentsline{toc}{chapter}{References}

\markboth{References}{References}

\phantomsection\label{refs}




\end{document}
